\documentclass[12pt]{article}

\usepackage{fancyhdr}
\usepackage{geometry}
\usepackage[square]{natbib}

\renewcommand{\bibsection}{\begin{flushleft}\textbf{\refname}\end{flushleft}}

\lhead{CS 7400 -- Principles of Programming Languages \\
  Project Proposal, due October 23, 2012}
\rhead{Phillip Mates, Erik Silkensen \\
  \texttt{\{mates,ejs\}@ccs.neu.edu}}

\begin{document}

\thispagestyle{fancy}

% understanding of the paper
Continuation Passing Style (CPS) is a popular intermediate representation form
in compilers that often requires several optimizations to overcome its
complexities. The 1993 paper ``The Essence of Compiling with Continuations''
by Flanagan et.\ al.\ shows the equivalence between CPS and A-Normal Form (ANF)
intermediate representations. The authors achieve this by creating a
translation between a CPS abstract machine and an ANF abstract machine. This
result demonstrates that the more efficient 1-pass ANF translation can be
used to achieve the same result as a 3-pass CPS translation.

Flanagan et.\ al.\ begin with a Core Scheme (CS), which they naively convert to
CPS using the $\mathcal{F}$ function [Fischer]. Administrative
$\overline{\lambda}$ terms introduced by $\mathcal{F}$ are then removed using a
$\overline{\beta}$-normalization function [Sabry]. At this point a $C_{cps}E$
abstract machine is built to operate over
$\overline{\beta}(\mathcal{F}(\textit{CS}))$. The authors proceed to define the
$C_{cps}EK$ machine, which removes some redundancy to do with \ldots.

% transformations
CS $\to$ CPS via $\mathcal{F}$ (naive CPS transformation) 

CPS $\to$ CPS(CS) via $\overline{\beta}$-normalization uses $C_{cps}E$

$C_{cps}EK$ removes some redundancy

CPS(CS) $\to$ A(CS) via $\mathcal{U}$ (un-CPS) uses $C_{a}EK$

% questions
In order to formalize their argument, Flanagan et.\ al.\ utilize techniques
widely accepted in the Programming Languages community, such as defining
semantics in terms of abstract machines and using evaluation contexts for
transformation reductions. On the surface these techniques seem intuitive, but
further inspection led us to realize our inability to reason about the models
in this paper by hand. ANF program transformation \ldots

% Tools/Proposal
Tools/Proposal

\bibliographystyle{abbrvnat}
\nocite{*}
\bibliography{refs}

\end{document}
