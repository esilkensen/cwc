\documentclass[11pt]{article}

\usepackage{fancyhdr}
\usepackage{geometry}
\usepackage{stmaryrd}
\usepackage[numbers]{natbib}

\renewcommand{\bibsection}{\begin{center}\textbf{\refname}\end{center}}

\lhead{CS 7400 -- Principles of Programming Languages \\
  Project Proposal, due October 23, 2012}
\rhead{Phillip Mates, Erik Silkensen \\
  \texttt{\{mates,ejs\}@ccs.neu.edu}}
\cfoot{}

\begin{document}

\thispagestyle{fancy}

% (1) Summarize your current understanding of the paper.
Compilers for higher-order languages often transform programs into the
continuation-passing style (CPS) as an intermediate representation. In the
language of CPS terms, all procedures take a continuation argument $k$
that represents the ``rest of the computation;'' procedures apply $k$ to their
return value instead of returning directly. This form is beneficial to
compilers because it enables optimizations by way of simple $\beta$- and
$\eta$-reductions, and it simplifies code generation by resembling a low-level
target language. However, it comes at a cost: a na\"{\i}ve CPS transformation
greatly expands a program's size, which a realistic compiler must work to
reduce.
% (and effectively invert).

\citet{Sabry:1992zr} observed that the transformation typically involves
three phases: compilers translate into CPS, simplify the result, and then
% effectively
invert the translation during code generation. They showed how
a set of so-called $A$-reductions applied directly to a source program yields a
result equivalent to the combined effect of the CPS transformation.
\citet{Flanagan:1993fk} clarified this correspondence by developing
a sequence of abstract machines simulating the compilation of a core
Scheme-like language using both CPS and $A$-reduction strategies.
% Scheme-like language
% ($\mathit{CS}$).
% , with and without continuations.
%In particular, they consider CEK machines for $\mathit{CS}$,
%$\mathit{CPS}(\mathit{CS})$, and $A(\mathit{CS})$ (the languages of
%$\mathit{CS}$ terms in CPS and $A$-normal form).
They discuss the details of the machines and the translations between the
languages, and prove the equivalence of CEK machines specialized for CPS and
$A$-normal form terms. 
%$\mathit{CPS}(\mathit{CS})$ and $A(\mathit{CS})$.
They conclude that compilers
using $A$-normal forms may enjoy the benefits of CPS-based compilers without
the overhead.

% (2) State which aspects of the paper you would like to understand in depth.
% (3) Explain which tools (Redex and otherwise) you wish to use to model the
% paper's insights so that you gain the desired insight.

While we understand their reasoning at a high level, we would like to grasp
the subtleties of the abstract machines and language transformations.
Towards that end, we plan to implement each machine and transformation
in Redex. By running executable models of the paper's mathematics on concrete
examples, we will gain a better understanding of how programs change throughout
the transformations, how modifications to the machines remove redundancies, and
the reasons for why it may be advantageous for a compiler to use $A$-reductions
instead of CPS.

As a side effect, we also think this project will help introduce us to how a
tool such as Redex is useful to programming language researchers in the
real world. For example, as we began implementing a model of the paper,
we noticed what seem to be two typos in the na\"{i}ve and inverse CPS
transformations. We were also puzzled by the definition of evaluation contexts
for the $A$-reductions, and plan to check our understanding by implementing a
working model in Redex.
%Along with a deeper comprehension of the paper, this project will serve as a re%al-world motivator towards the use of lightweight mechanization tools such as Redex.
%This project should also introduce us to some of the real-world motivation
%behind the Redex tool as a form of lightweight
%mechanization~\cite{Klein:2012uq}.
%We started implementing the models from the
%paper by typing them directly into Redex, which revealed potential
%typos in the paper's na\"{i}ve and inverse CPS transformations.
%With complete Redex models, these typos and other suspicions, such as the incom%pleteness of the $A$-reductions' evaluation context definition, can be validated.
%
% Paragraph about A-reductions vs beta-reductions
% While introducing A-reductions on the source language, \citet{Flanagan:1993fk}
% mention that the corresponding reduction on CPS terms is the
% $\overline{\beta}$-reduction, which removes administrative $\lambda$'s from the
% na\"{i}ve CPS translation.  In order to better understand the relation between
% A and $\overline{\beta}$ reductions, we propose to model both reductions in
% Redex.  Doing so is especially interesting since as they appear in Figure 7
% \cite{Flanagan:1993fk}, the A-reduction and evaluation context are either
% non-intuitive or in fact contain errors.

% % Paragraph about errors in other figures
% % Errors in Figure 3 and Figure 6.
% To this end, we have identified two other small potential errors in the figures
% of \cite{Flanagan:1993fk}.  The last rule of $\Phi$ in the CPS transformation
% figure, $\Phi \llbracket \lambda x_1 \ldots x_n . M \rrbracket = \lambda k x_1
% \ldots x_n . \mathcal{F} \llbracket M \rrbracket$, doesn't seem to use the
% newly introduced $k$ argument in the body of the $\lambda$.  In the second
% potential error, the last rule of $\Psi$ in the inverse CPS transformation
% figure, $\Psi \llbracket \lambda k x_1 \ldots x_n . P \rrbracket = \lambda x_1
% \ldots x_n . \mathcal{U} \llbracket M \rrbracket$, introduces an unspecified
% $M$ on the right hand side. As a means to resolve these suspicions and build
% our confidence in the correctness of the other figures presented, we propose to
% model the abstract machines (figures 2,4,5, and 8) and CPS transformations
% (figures 3 and 6) from \cite{Flanagan:1993fk} in Redex.

% % Paragraph about "Running Your Research"
% % TODO: finish this paragraph:
% In addition to finding errors, building executable models of the paper's
% mathematics will enable us to run the mathematics on concrete examples. \ldots
% solidify our understanding of how code shrinks and grows through
% transformations, as well as, what redundancies introduced and removed via
% language transformations and abstract machine formulations.

% \pagebreak

\bibliographystyle{abbrvnat}
\bibliography{refs}

\end{document}
