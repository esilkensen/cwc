\documentclass[12pt]{article}
\usepackage{url,afterpage,enumerate,amssymb,amsmath,graphicx,subfigure,tabularx,array,natbib,caption}
\usepackage[letterpaper,margin=1.5cm]{geometry}
\setlength{\parskip}{1ex} %--skip lines between paragraphs
\setlength{\parindent}{0pt} %--don't indent paragraphs


%-- Commands for header
\renewcommand{\title}[1]{\textbf{#1}\\}
\renewcommand{\line}{\begin{tabularx}{\textwidth}{X>{\raggedleft}X}\hline\\\end{tabularx}\\[-0.5cm]}
\newcommand{\leftright}[2]{\begin{tabularx}{\textwidth}{X>{\raggedleft}X}#1
\newcommand{\norm}[1]{\lVert#1\rVert}
& #2\\\end{tabularx}\\[-0.5cm]}

%-- definitions
\newtheorem{my_def}{Definition}[section]

\begin{document}

\title{Project Proposal: The Essence of Compiling with Continuations}
\line
\leftright{\today}{CS 7400} %-- left and right positions in the header
\leftright{Phillip Mates and Erik Silkensen}{} \\ %-- left and right positions in the header

% A first reading of a technically complex paper should leave you with a
% shallow understanding and a desire to understand certain aspects in more
% depth. The best way to express this desire is to articulate your
% understanding and to phrase questions in this context. These questions tend
% to tell you which "tools" (mathematical techniques, implementation,
% measurements, etc) you will need to use to find answers.

% understanding of the paper
Continuation Passing Style (CPS) is a popular intermediate representation form in compilers that often requires several optimizations to overcome its complexities. The 1993 paper ``The Essence of Compiling with Continuations'' by Flanagan et.\ al.\ shows the equivalence between CPS and A-Normal Form (ANF) intermediate representations.
The authors achieve this by creating a translation between a CPS abstract machine and an ANF abstract machine.
This result demonstrates that the more efficient 1-pass ANF translation can be used to achieve the same result as a 3-pass CPS translation.

Flanagan et.\ al.\ begin with a Core Scheme (CS), which they naively convert to CPS using the $\mathcal{F}$ function [Fischer]. Administrative $\overline{\lambda}$ terms introduced by $\mathcal{F}$ are then removed using a $\overline{\beta}$-normalization function [Sabry]. At this point a $C_{cps}E$ abstract machine is built to operate over $\overline{\beta}(\mathcal{F}(\textit{CS}))$. The authors proceed to define the $C_{cps}EK$ machine, which removes some redundancy to do with \ldots.

% transformations
CS $\to$ CPS via $\mathcal{F}$ (naive CPS transformation) 

CPS $\to$ CPS(CS) via $\overline{\beta}$-normalization uses $C_{cps}E$

$C_{cps}EK$ removes some redundancy

CPS(CS) $\to$ A(CS) via $\mathcal{U}$ (un-CPS) uses $C_{a}EK$

% questions
In order to formalize their argument, Flanagan et.\ al.\ utilize techniques widely accepted in the Programming Languages community, such as defining semantics in terms of abstract machines and using evaluation contexts for transformation reductions.
On the surface these techniques seem intuitive, but further inspection led us to realize our inability to reason about the models in this paper by hand.
ANF program transformation \ldots

% Tools/Proposal
Tools/Proposal

% Write up a one-page memo in journal-paper style that
% \begin{itemize}
%   \item summarizes your current understanding of the paper
%   \item states which aspects of the paper you would like to understand in depth
%   \item explains which tools (Redex and otherwise) you wish to use to model the
%     paper's insights so that you gain the desired insight. Researchers tend to
%     think of this memo as a proposal.
% \end{itemize}

\end{document}
